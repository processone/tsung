%%% user_manual.tex --- 

%% Author: XX.YY@IDEALX.com
%% Version: $Id$

\def\RCS$#1: #2 ${\expandafter\def\csname RCS#1\endcsname{#2}}
\RCS$Revision$
\RCS$Date$

\documentclass{IDXDOC-en}
\usepackage{shortcuts}

% --------------------------------------------
% Title 
% --------------------------------------------
\doctitle{Tsung User's manual}

%\addauthor{Nicolas}{Niclausse}{nicolas.niclausse@niclux.org}

\doccopyright{IDEALX S.A.S.}
\docversion{\RCSRevision}
\docreldate{\RCSDate}
\docref{tsung-user-manual}

\begin{document}
\maketitle
\Abstract{Tsung user's manual}
\newpage
\tableofcontents

\section{Introduction}

\subsection{What is Tsung ?}

\program{Tsung} (formerly IDX-Tsunami) is a distributed load testing tool. It is
protocol-independent and can currently be used to stress HTTP, SOAP
PostgreSQL, and Jabber servers.

It is distributed under the GNU General Public License version 2.

\subsection{What is Erlang and why is it important for Tsung ?}

\program{Tsung's} main strength is its ability to simulate a huge number of
simultaneous user from a single CPU. When used on cluster you can
generate a really impressive load on a server with a modest cluster,
easy to set-up and to maintain.

\program{Tsung} is developed in Erlang and this is where the power
of \program{Tsung} resides.

\par Erlang is a \emph{concurrency-oriented} programming language.
Tsung is based on the Erlang OTP (Open Transaction Platform) and
inherits several characteristics from Erlang: 

\begin{itemize}
\item \emph{Performance}: Erlang has been made to support hundred thousands
of lightweight processes in a single virtual machine. 
\item \emph{Scalability}: Erlang runtime environment is naturally
distributed, promoting the idea of process's location transparency. 
\item \emph{Fault-tolerance}:Erlang has been built to develop robust,
fault-tolerant systems. As such, wrong answer sent from the server
to \program{Tsung} does not make the whole running benchmark crash. 
\end{itemize}

More informations on Erlang on \url{http://www.erlang.org} and
\url{http://www.erlang-projects.org/}


\subsection{Tsung background}

History:
\begin{itemize}
\item \program{Tsung} is being developed since 2001 (it was called
  IDX-Tsunami at this time)
\item It is an industrial strength implementation of a \emph{stochastic model}
for real users simulation. User events distribution is based on a
Poisson Process. More information on this topic in:

Z. Liu, N. Niclausse, et C. Jalpa-Villanueva.  \strong{Traffic Model
  and Performance Evaluation of Web Servers}. \emph{Performance Evaluation,
Volume 46, Issue 2-3, October 2001}.

\item This model has already been tested in the INRIA \emph{WAGON}
  research prototype (Web trAffic GeneratOr and beNchmark). WAGON was
  used in the \url{http://www.vthd.org/} project (Very High Broadband
  IP/WDM test platform for new generation Internet applications, 2000-2004).

\end{itemize}

\program{Tsung} has been used for very high load tests:
 
\begin{itemize}
\item \emph{Jabber} protocol: 10 000 simultaneous users.
  \program{Tsung} was running on a 3-computers cluster (CPU
  800Mhz)
\item \emph{HTTP and HTTPS} protocol: 12 000 simultaneous users.
  \program{Tsung} were running on a 4-computers cluster. The
  tested platform reached 3 000 requests per second.
\end{itemize}

\program{Tsung} has been used  at: 

\begin{itemize}
\item \emph{DGI} (Direction G�n�rale des imp�ts): French finance ministry 
\item \emph{Cap Gemini Ernst \& Young}
\item \emph{IFP} (Institut Fran�ais du P�trole): French Research Organization
for Petroleum 
\item \emph{LibertySurf}
\end{itemize}

\section{Features}

\subsection{Tsung main features}

\begin{itemize}
\item \emph{High Performance}: \program{Tsung} can simulate a
  huge number of simultaneous users per physical computer: It can
  simulates thousands of users on a single CPU (Note: a simulated user
  is not always active: it can be idle during a \varname{thinktime}
  period). Traditional injection tools can hardly go further than a
  few hundreds (Hint: if all you want to do is requesting a single URL
  in a loop, use \program{ab}; but if you want to build complex
  scenarios with extended reports, \program{Tsung} is for you).
\item \emph{Distributed}: the load can be distributed on a cluster of
client machines 
\item \emph{Multi-Protocols} using a plug-in system: HTTP (both standard
web traffic and SOAP) and Jabber are currently supported. LDAP and
SMTP are on the TODO list. 
\item \emph{SSL} support 
\item \emph{Several IP addresses} can be used on a single machine using
the underlying OS IP Aliasing 
\item \emph{OS monitoring} (CPU, memory and network traffic) using Erlang
agents on remote servers or \emph{SNMP}
\item \emph{XML configuration system}: complex user's scenarios are written
  in XML. Scenarios can be written with a simple browser using the
  Tsung recorder (for HTTP only).
\item \emph{Dynamic scenarios}: You can get dynamic data from the
  server under load (without writing any code) and re-inject it in
  subsequent requests.
\item \emph{Mixed behaviours}: several sessions can be used to simulate
different type of users during the same benchmark. You can define
the proportion of the various behaviours in the benchmark scenario. 
\item \emph{Stochastic processes}: in order to generate a realistic traffic,
user thinktimes and the arrival rate can be randomize using a probability
distribution (currently exponential) 
\end{itemize}

\subsection{HTTP related features}

\par 


\begin{itemize}
\item HTTP/1.0 and HTTP/1.1 support
\item GET and POST requests 
\item Cookies: Automatic cookies management 
\item \verb|'|GET If-modified since\verb|'| type of request 
\item WWW-authentication Basic 
\item Proxy mode to record sessions using a Web browser 
\item SOAP support using the HTTP mode (the SOAPAction HTTP header is
  handled).
\item HTTP server or proxy server load testing.
\end{itemize}

\subsection{Jabber related features}

\begin{itemize}
\item Authentication, presence and register messages 
\item Chat messages to online or offline users 
\item Roster set and get requests 
\item Global users\verb|'| synchronization can be set on specific actions
\item raw XML messages
\end{itemize}

\subsection{PostgreSQL related features}

This pluging is still experimental

\begin{itemize}
\item Basic Authentication
\item Basic Queries (Extended queries not yet supported)
\item Proxy mode to record sessions
\end{itemize}

\subsection{Complete reports set}

Measures and statistics produced by Tsung are extremely feature-full.
They are all represented as a graphic. \program{Tsung} produces
statistics regarding:

\begin{itemize}
\item \emph{Performance}: response time, connection time, decomposition
of the user scenario based on request grouping instruction (called \textit{transactions}), requests
per second 
\item \emph{Errors}: Statistics on page return code to trace errors 
\item \emph{Target server behaviour}: An Erlang agent can gather information
from the target server(s). Tsung produces graphs for CPU and memory
consumption and network traffic. SNMP is also supported. 
\end{itemize}
\par Note that \program{Tsung} takes care of the synchronization process
by itself. Gathered statistics are �synchronized�.

 It is possible to generate graphs during the benchmark as statistics
are gathered in real-time.

\subsection{Highlights}

\program{Tsung} has several advantages over other injection tools: 


\begin{itemize}
\item \emph{High performance} and \emph{distributed benchmark}: You
  can use Tsung to simulate tens of thousands of virtual users.
\item \emph{Ease of use}: The hard work is already done for all supported
protocol. No need to write complex scripts. Dynamic scenarios only
requires small trivial piece of code.
% Tsung scenarii realisation is mostly based on 
\item \emph{Multi-protocol support}: \program{Tsung} is for example one of
the only tool to benchmark SOAP applications 
\item \emph{Monitoring} of the target server(s) to analyze the behaviour
and find bottlenecks. For example, it has been used to analyze cluster
symmetry (is the load properly balanced ?) and to determine the best
combination of machines on the three cluster tiers (Web engine, EJB
engine and database) 
\end{itemize}



\section{Installation}

This package has been tested on Linux, FreeBSD and Solaris. It should
work on Erlang supported platforms (Linux, Solaris, *BSD, Win32 and
MacOS-X).

\subsection{Dependencies}
\begin{itemize}
\item Erlang/OTP R9C-0 and up (included R10B-8)
  (\url{http://www.erlang.org/download.html}).  RedHat users can
  download a R9C-2 rpm at
  \url{http://www.erlang-projects.org/Public/rpmdeb/rpm_erlang_otp_r9c-2/view}.
\item xmerl-0.19 (included in Erlang R10B) (\url{http://sowap.sourceforge.net/download.html}). Debian and Redhat
    binaries are provided at
    \url{http://tsung.erlang-projects.org/dist/}
  \item extended regexp module (used for dynamic variables):
    gregexp.erl available at
    \url{http://www.cellicium.com/erlang/contribs/} . The module is
    included in the source and binary distribution of \program{Tsung}. It
    is released under the EPL License.
  \item pgsql module made by Christian Sunesson (for the postgresql plugin):
    sources available at
    \url{http://jungerl.sourceforge.net/} . The module is
    included in the source and binary distribution of \program{Tsung}. 
   \item  gnuplot and perl5 (optional; for graphical output with
    \command{tsung\_stats.pl} script).  The Template Toolkit is used for HTML
    reports (see \url{http://template-toolkit.org/})
  \item for distributed tests, you need an ssh access to remote
    machines without password (use a RSA/DSA key without pass-phrase or
    ssh-agent) (rsh is also supported)
\item bash
\end{itemize}
\subsection{Compilation}

\begin{Verbatim}
./configure
 make
 make install
\end{Verbatim}
\subsection{Configuration}

The default configuration file is \file{~/.tsung/tsung.xml} (
there is a sample file
\file{/usr/share/doc/tsung/examples/tsung.xml}).

Log files are saved in \file{~/.tsung/log/} . A new subdirectory
is created for each test using the current date as name
(\file{~/.tsung/log/20040217-09:40} for ex.)

\subsection{Feedback}

Use the Tsung mailing list (see
\url{http://lists.idealx.org/info/idx-tsunami}) if you have
suggestions or questions about \program{Tsung}.

\section{HTTP benchmark approach}

\subsection{benchmarking a Web server}

\begin{enumerate}
\item Record one or more sessions: start the recorder with: \command{tsung
    recorder}, and then configure your browser to use Tsung
  proxy recorder (the listen port is 8090). A session file will be
  created. For HTTPS recording, use \userinput{http://\{} instead of
    \userinput{https://} in your browser.
\item Edit / organize scenario, by adding recorded sessions in the
  configuration file.
\item Write small code for dynamic parts if needed and place dynamic mark-up
in the scenario.
\item Test and adjust scenario to have a nice progression of the load. This
is highly dependent of the application and of the size of the target
server(s). Calculate the normal duration of the scenario and use the
interarrival time between users and the duration of the phase to estimate
the number of simultaneous users for each given phase. 
\item Launch benchmark with your first application parameters set-up:
  \command{tsung start}
\item Wait for the end of the test or stop by hand with
  \command{tsung stop} (reports can also be generated during the
  test (see � \ref{sec:statistics-reports}) : the statistics are
  updated every 10 seconds). For a brief summary of the current
  activity, use \command{tsung status}
\item Analyze results, change parameters and relaunch another benchmark 
\end{enumerate}

\subsection{benchmarking a proxy server}

By default, the HTTP plugin is used to benchmark HTTP servers. But you
can also benchmark HTTP Proxy server. To do that, you must add in the
\varname{options} section:

\begin{Verbatim}
  <option type="ts_http" name="http_use_server_as_proxy" value="true"></option>
\end{Verbatim}

\section{PostgreSQL benchmark approach}

It's the same approach as HTTP: first you start to record one or more
sessions with the recorder: 
\command{idx-tsunami -p pgsql  recorder}

This will start a proxy listening to port 8090 and will proxy requests
to 127.0.0.0:5432.

\section{Jabber benchmark approach}

This paragraph explains how to write a session for Jabber.

There are two differences between HTTP and Jabber testing:
\begin{enumerate}
\item There is no recorder for Jabber, so you have to write your
  sessions by hand (an example is provided in
  \ref{sec:sessions:jabber}).
\item the jabber plugin does not parse XML; instead it uses packet
  acknowledgments. The rest of this paragraph will explain this
  feature.
\end{enumerate}

Since the jabber plugin does not parse XML (historically, it was for
performance reasons), and also due to the bidirectional nature of the
jabber protocol, you must have a way to tell when a request is
finished. There are 3 possibilities:

\begin{description}
 \item[ack=local] as soon as a packet is received from the server, the
request is considered as completed. Hence if you use a local ack with a request
that do not require a response from the server (presence for ex.), it
 will wait forever (or until a timeout is reached).
 \item[ack=no\_ack] as soon as the request is send, it is considered as completed (do
not wait for incoming data)
 \item[ack=global] synchronized users. its main use is for waiting for all
users to connect before sending messages. To do that, set a request
with global ack (it can be the first presence msg:

\begin{Verbatim}
   <request> <jabber type="presence" ack="global"/> </request>
\end{Verbatim}
)

You also have to specify the number of users to be connected:

\begin{Verbatim}
<option type="ts_jabber" name="global_number" value="100"></option>
\end{Verbatim}

To be sure that exactly \varname{global\_number} users are started, add the
\userinput{'maxnumber'} attribute to \varname{'users'}

\begin{Verbatim}
    <users maxnumber="100" interarrival="1.0" unit="second"></users>
\end{Verbatim}

If you do not specify maxnumber, the global ack will be reset every
\varname{global\_number} users
\end{description}

\section{Understanding tsung.xml configuration file}

\subsection{File structure}

 Scenarios are enclosed into Tsung tags:

 \begin{Verbatim}
<?xml version="1.0"?>
<!DOCTYPE tsung SYSTEM "/usr/share/tsung/tsung-1.0.dtd" [] >
<tsung loglevel="info">
...
</tsung>
 \end{Verbatim}

 If you add the attribute \userinput{dumptraffic="true"}, all the
 traffic will be logged to a file. \emph{Warn:} this will considerably
 slow down Tsung, so use with care. It is useful for debugging
 purpose.

\subsection{Clients and server}

 Scenarios start with clients (Tsung cluster) and server definitions:

For non distributed load, you can use a basic setup like:

\begin{Verbatim}
  <clients>
     <client host="localhost" use_controller_vm="true"/>
  </clients>

  <server host="192.168.1.1" port="80" type="tcp"></server>
 \end{Verbatim}

This will start the load on the same host and on the same Erlang
virtual machine as the controller.

The server is the entry point into the cluster (only one server
 can be defined).

The next example is a more complex, and use several features for
advanced distributed testing:

\begin{Verbatim}
  <clients>
     <client host="louxor" weight="1" maxusers="800">
         <ip value="10.9.195.12"></ip>
         <ip value="10.9.195.13"></ip>
     </client>
     <client host="memphis" weight="3" maxusers="600" cpu="2">
         <ip value="10.9.195.14"></ip>
     </client>
  </clients>

  <server host="10.9.195.1" port="8080" type="tcp"></server>
 \end{Verbatim}

 
 Several virtual IP can be used to simulate more machines. This is
 very useful when a load-balancer use the client\verb|'|s IP to
 distribute the traffic among a cluster of servers. \strong{New in
   1.1.1:} IP is not mandatory. If not specified, the default IP will
 be used.
 
 In this example, a second machine is used in the Tsung cluster,
 with a higher weight, and 2 cpus. Two Erlang virtual machines will be
 used to take advantage of the number of CPU.

 By default, the load is distributed uniformly on all CPU (one cpu
 per client by default). The weight parameter (integer) can be used to
 take into account the speed of the client machine. For instance, if
 one real client has a weight of 1 and the other client has a weight
 of 2, the second one will start twice the number of users as the
 first (the proportions will be 1/3 and 2/3). In the earlier example
 where for the second client has 2 CPU and weight=3, the weight is
 equal to 1.5 for each CPU.


\subsection{Monitoring}

\par Scenarios can contain optional monitoring informations. For example,
here is a cluster monitoring definition based on Erlang agents, for
a cluster of 6 computers:

\begin{Verbatim}
  <monitoring>
    <monitor host="geronimo" type="erlang"></monitor>
    <monitor host="bigfoot-1" type="erlang"></monitor>
    <monitor host="bigfoot-2" type="erlang"></monitor>
    <monitor host="f14-1" type="erlang"></monitor>
    <monitor host="f14-2" type="erlang"></monitor>
    <monitor host="db" type="erlang"></monitor>
  </monitoring>
\end{Verbatim}

The type keyword snmp can replace the erlang keyword, if SNMP monitoring
is preferred. They can be mixed. erlang is the default value for
monitoring.

\par Note: For Erlang monitoring, monitored computers need to be
accessible through the network. SSH (or rsh) needs to be configured to
allow connection without password on. \strong{You must use the same
  version of Erlang/OTP on all nodes otherwise it may not work
  properly !}

\subsection{Defining the load progression}

\par The load progression is set-up by defining several arrival phases:

\begin{Verbatim}
 <load>
  <arrivalphase phase="1" duration="10" unit="minute">
    <users interarrival="2" unit="second"> </users>
  </arrivalphase>

  <arrivalphase phase="2" duration="10" unit="minute">
    <users interarrival="1" unit="second"> </users>
  </arrivalphase>

  <arrivalphase phase="3" duration="10" unit="minute">
    <users interarrival="0.1" unit="second"> </users>
  </arrivalphase>
 </load>
\end{Verbatim}

With this setup, during the first 10 minutes of the test, a new user
will be created every 2 seconds, then during the next 10 minutes, a
new user will be created every second, and for the last 10 minutes,
10 users will be generated every second. The test will finish when
all users have ended their session.

The load generated in terms of HTTP requests / seconds will also
depend on the mean number of requests within a session (if you have a
mean value of 100 requests per session and 10 new users per seconds,
the theoretical average throughput will be 1000 requests/ sec).
 
\subsection{Setting options}

\par Default values can be set-up globally: thinktime between requests
in the scenario and ssl cipher algorithms. These values overrides
those set in session configuration tags if override is true.
\begin{Verbatim}
  <option name="thinktime" value="3" random="false" override="true"/>
  <option name="ssl_ciphers" 
           value="EXP1024-RC4-SHA,EDH-RSA-DES-CBC3-SHA"/>
\end{Verbatim}

\subsubsection{Jabber options}

Default values for specific protocols can be defined. Here is an
example of option values for Jabber:

\begin{Verbatim}
  <option type="ts_jabber" name="global_number" value="5" />
  <option type="ts_jabber" name="userid_max" value="100" />
  <option type="ts_jabber" name="domain" value="jabber.org" />
  <option type="ts_jabber" name="username" value="myuser" />
  <option type="ts_jabber" name="passwd" value="mypasswd" />
\end{Verbatim}

Using these values, users will be \userinput{myuserXXX} where XXX is an integer in
the interval [1:userid\_max] and passwd  \userinput{mypasswdXXX}

If not set in the configuration file, the values will be set to:
\begin{itemize}
\item global\_number = FIXME
\item userid\_max  = FIXME
\item domain = FIXME
\item username = FIXME
\item passwd = FIXME
\end{itemize}

\subsubsection{HTTP options}

For HTTP, you can set the \varname{UserAgent} values
(\string{available since Tsung 1.1.0}), using a probability for each
value (the sum of all probabilities must be equal to 100)

\begin{Verbatim}
  <option type="ts_http" name="user_agent">
    <user_agent probability="80">
       Mozilla/5.0 (X11; U; Linux i686; en-US; rv:1.7.8) Gecko/20050513 Galeon/1.3.21
    </user_agent>
    <user_agent probability="20">
      Mozilla/5.0 (Windows; U; Windows NT 5.2; fr-FR; rv:1.7.8) Gecko/20050511 Firefox/1.0.4
    </user_agent>
  </option>
\end{Verbatim}

\subsection{Sessions}

\subsubsection{Overview and examples}

Sessions define the content of the scenario itself. They describe
the requests to execute.

This example shows several features of the HTTP protocol support in
Tsung: GET and POST request, basic authentication, transaction for
statistics definition, ...


\begin{Verbatim}
<sessions>
  <session name="http-example" popularity="70" type="ts_http">

    <request> <http url="/" method="GET" version="1.1">
                    </http> </request>
    <request> <http url="/images/logo.gif"
               method="GET" version="1.1" 
               if_modified_since="Fri, 14 Nov 2003 02:43:31 GMT">
              </http></request>

    <thinktime value="20" random="true"></thinktime>

    <transaction name="index_request">
     <request><http url="/index.en.html"
                          method="GET" version="1.1" >
              </http> </request>
     <request><http url="/images/header.gif"
                          method="GET" version="1.1">
              </http> </request>
    </transaction>

    <thinktime value="60" random="true"></thinktime>
    <request>
      <http url="/" method="POST" version="1.1"
               contents="bla=blu">
      </http> </request>
    <request>
       <http url="/bla" method="GET" version="1.1"
             contents="bla=blu&name=glop">
       <www_authenticate userid="Aladdin"
                         passwd="open sesame"/></http>
    </request>
  </session>

  <session name="backoffice" popularity="30" ...>
  ... </session>
</sessions>
\end{Verbatim}

Each session has a given probability. This is used to decide which
session a new user will execute. The sum of all session\verb|'|s
probabilities must be 100.

A transaction is just a way to have customized statistics. Say if you
want to know the response time of the login page of your website, you
just have to put all the requests of this page (HTML + embedded
pictures) within a transaction. In the example above, the transaction
called \varname{index\_request} will gives you in the
statistics/reports the mean response time to get
\userinput{index.en.html + header.gif}. Be warn that If you have a
thinktime inside the transaction, the thinktime will be part of the
response time.

\label{sec:sessions:jabber}
\par Here is an example of a session definition for the Jabber protocol:
\begin{Verbatim}
<sessions>
 <session probability="70" name="jabber-example" type="ts_jabber">

    <request> <jabber type="connect" ack="no_ack" /> </request>

    <thinktime value="2"></thinktime>

    <transaction name="authenticate">
      <request> <jabber type="authenticate" ack="local"> </jabber> </request>
    </transaction>

    <request> <jabber type="presence" ack="no_ack"/> </request>

    <thinktime value="2"></thinktime>

    <transaction name="roster">
      <request><jabber type="iq:roster:get" ack="local"> </jabber> </request>
    </transaction>

    <thinktime value="30"></thinktime>

    <transaction name="online">
    <request> <jabber type="chat" ack="no_ack" size="16" destination="online"/></request>
    </transaction>
    <thinktime value="30"></thinktime>

    <transaction name="offline">
      <request> <jabber type="chat" ack="no_ack" size="56" destination="offline"/><request>
    </transaction>

    <thinktime value="30"></thinktime>

    <transaction name="close">
      <request> <jabber type="close" ack="local"> </jabber></request>
    </transaction>
  </session>
</sessions>
\end{Verbatim}

\subsubsection{Dynamic substitutions}

 Dynamic substitution are mark-up placed in element of the scenario.
For HTTP, this mark-up can be placed in basic authentication (www\_authenticate
tag: userid and passwd attributes), URL (to change GET parameter)
and POST content.

Those mark-up are of the form \userinput{\%\%Module:Function\%\%}.
Substitutions are executed on a request-by-request basis, only if the
request tag has the attribute \userinput{subst="true"}.

When a substitution is requested, the substitution mark-up is replaced by
the result of the call to the Erlang function:
\userinput{Module:Function({Pid, DynData})} where Pid is the Erlang process
id of the current virtual user and DynData the list of all Dynamic
variables (\strong{Warn: before version 1.1.0, the argument was just the
  Pid !}).

Here is an example of use of substitution in a Tsung scenario:

\begin{Verbatim}
<session name="rec20040316-08:47" probability="100" type="ts_http">
 <request subst="true">
  <http url="/echo?symbol=%%symbol:new%%" method="GET">
  </http></request>
</session>
\end{Verbatim}

Here is the Erlang code of the module used for dynamic substitution:

\begin{Verbatim}
-module(symbol).
-export([new/1]).

new({Pid, DynData}) ->
    case random:uniform(3) of
        1 -> "IBM";
        2 -> "MSFT";
        3 -> "RHAT"
    end.
\end{Verbatim}

(use \command{erlc} to compiled the code, and put the resulting .beam
file in \file{$PREFIX/lib/erlang/lib/tsung-X.X.X/ebin/} on all client
machines)

As you can see, writing scenario with dynamic substitution is trivial.

If you want to set unique id, you can use the built-in function
\varname{ts\_user\_server:get\_unique\_id}.
\begin{Verbatim}
<session name="rec20040316-08:47" probability="100" type="ts_http">
 <request subst="true">
  <http url="/echo?id=%%ts_user_server:get_unique_id%%" method="GET">
  </http></request>
</session>
\end{Verbatim}

\subsubsection{Reading external file}
\strong{New in 1.0.3}: A new  module \varname{ts\_file\_server} is available. You
can use it to read external files. For example, if you need to read user
names and passwd from a csv file, you can do it with it (currently,
you can read only a single file).

You have to add this in the xml configuration file:
\begin{Verbatim}
 <option name="file_server"  value="/tmp/userlist.csv"></option>
\end{Verbatim}

Now you can build you own function to use it, for example:

\begin{Verbatim}
-module(readcsv).
-export([user/1]).

user(Pid)->
    {ok,Line} = ts_file_server:get_next_line(), 
    [Username, Passwd] = string:tokens(Line,";"),
    "username=" ++ Username ++"&amp;passwd=" ++ Passwd.
\end{Verbatim}


In your session, use something like: 

\begin{Verbatim}
  <request subst="true">
    <http url='/login.cgi' version='1.0' contents='%%readcsv:user%%&amp;op=login'
    content_type='application/x-www-form-urlencoded' method='POST'>
    </http>
  </request>
\end{Verbatim}


Two functions are available: \varname{ts\_file\_server:get\_next\_line}
and \varname{ts\_file\_server:get\_random\_line}. For the
\varname{get\_next\_line} function, when the end of file is reached, the
first line of the file will be the next line.


\subsubsection{Dynamic variables}

In some cases, you may want to use a value given by the server in a
response later in the session, and this value is \strong{dynamically
generated} by the server for each user. For this, you can use
\userinput{<dyn\_variable>} in the scenario

Let's take an example with HTTP. You can easily grab a value in a HTML
form like:
\begin{Verbatim}
<form action="go.cgi" method="POST">
<hidden name="random_num" value="42"></form>
</form>
\end{Verbatim}

with:
\begin{Verbatim}
 <request>
   <http url="/testtsung.html" method="GET" version="1.0"></http>
   <dyn_variable name="random_num" ></dyn_variable>
 </request>
\end{Verbatim}

Now \varname{random\_num} will be set to 42 during the user's session. It's
value will be replace in all mark-up of the form
\userinput{\%\%\_random\_num\%\%} if and only if the \varname{request} tag has the
attribute \userinput{subst="true"}, like:

\begin{Verbatim}
    <request subst="true">
      <http url='/go.cgi' version='1.0' 
      contents='username=nic&amp;random_num=%%_random_num%%&amp;op=login' 
      content_type='application/x-www-form-urlencoded' method='POST'>
      </http>
    </request>
\end{Verbatim}
  
If the dynamic value is not a form variable, you can set a regexp by
hand, for example to get the title of a HTML page:
\begin{Verbatim}
    <request>
      <http url="/testtsung.html" method="GET" version="1.0"></http>
      <dyn_variable name="mytitlevar" 
                    regexp="&lt;title&gt;\(.*\)&lt;/title&gt;"/>
    </request>
\end{Verbatim}

\subsubsection{Checking the server's response}

With the tag \varname{match} in a \varname{request} tag, you can check
the server's response against a given string, and do some actions
depending on the result. In any case, if it matches, this will
increment the \varname{match} counter, if it does not match, the
\varname{nomatch} counter will be incremented.

For example, let's say you want to test a login page. If the login is
ok, the server will respond with \computeroutput{Welcome !} in the
HTML body, otherwise not. To check that:
\begin{Verbatim}
 <request>
      <match do="continue" when="match">Welcome !</match>
      <http url='/login.php' version='1.0' method='POST' 
       contents='username=nic&amp;user_password=sesame'
       content_type='application/x-www-form-urlencoded' >
 </request>
\end{Verbatim}

You can use a regexp instead of a simple string.

The list of available actions to do is:
\begin{itemize}
\item continue
\item abort : abort the session
\item restart: restart the session. The maximum number of
  restarts is 3 by default.
\item loop: repeat the request, after 5 seconds. The maximum number of
  loops is 20 by default.
\end{itemize}

You can mixed several match tag in a single request:
\begin{Verbatim}
 <request>
      <match do="loop" sleep_loop="5" max_loop="10" when="match">Retry</match>
      <match do="abort" when="match">Error</match>
      <http url='/index.php' method=GET'>
 </request>
\end{Verbatim}

You can also do the action on "nomatch" instead of "match".

\section{Statistics and reports}
\label{sec:statistics-reports}

\subsection{Available stats}

\begin{itemize}
\item  request (response time for each request)
\item  page (response time for each set of requests (a page is a group
  of request not separated by a thinktime))
\item  connect (duration of the connection establishment)
\item  reconnect (number of reconnection)
\item  size\_rcv (size of responses)
\item  size\_sent (size of requests)
\item  session (duration of a user's session)
\item  users (number of simultaneous users)
\item  custom transactions
\end{itemize}

The mean response time (for requests, page, etc.) is computed every 10
sec (and reset). That's why you have the highest mean and lowest mean
values in the Stats report (the mean for the whole test is not
computed, since it's not very meaningful if you have several phases
with different input rates in your setup).

HTTP specific stats:
\begin{itemize}
\item counter for each response status (200, 404, etc.)
\end{itemize}


\subsection{Design}

A bit of explanation on the design and internals of the statistics engine:

Tsung was designed to handle thousands of requests/sec, for very
long period of times (several hours) so it do not write all data to
the disk (for performance reasons). Instead it computes on the fly an
estimation of the mean and standard variation for each type of data,
and writes these estimations every 10 seconds to the disk (and then
starts a new estimation for the next 10 sec). These computations are
done for two kinds of data:
\begin{itemize}
\item \varname{sample}, for things like response time
\item \varname{sample\_counter} when the input is a cumulative one (number of
packet sent for ex.).
\end{itemize}

There are also two other types of useful data (no averaging is done for
those) :
\begin{itemize}
\item \varname{counter}: a simple counter, for HTTP status code for ex.
\item \varname{sum} for ex. the cumulative HTTP response's size (it gives an
estimated bandwidth usage).
\end{itemize}



\subsection{Generating the report}

cd to the log directory of your test (say
\file{~/.tsung/log/20040325-16:33/}) and use the script
\command{tsung\_stats.pl}:

\begin{Verbatim}
/usr/lib/tsung/bin/tsung_stats.pl --html --extra --plot  
\end{Verbatim}

(use \userinput{--help} to view all available options)

\subsection{tsung summary}
\begin{figure}[htb]
  \begin{center}
    \includegraphics[width=0.6\linewidth]{tsung-report}
    \end{center}
      \caption{Report}
    \label{fig:report}
\end{figure}

\subsection{Graphical overview}


\begin{figure}[htb]
  \begin{center}
    \includegraphics[width=0.6\linewidth]{tsung-graph}
    \end{center}
      \caption{Graphical output}
    \label{fig:graph}
\end{figure}

%\section{Roadmap}

%FIXME

\section{References}

\begin{itemize}
\item \program{Tsung} home page: \url{http://tsung.idealx.org/}
\item \program{Tsung} description (French)\footnote{\url{http://www.erlang-projects.org/Members/mremond/events/dossier_de_presentat/block_10766817551485/file}}
\item Erlang web site \url{http://www.erlang.org/}
\item Erlang programmation, Micka�l R�mond, Editions Eyrolles, 2003
  \footnote{\url{http://www.editions-eyrolles.com/php.accueil/Ouvrages/ouvrage.php3?ouv_ean13=9782212110791}}
\item \emph{Making reliable system in presence of software errors}, Doctoral Thesis,
Joe Armstrong, Stockholm, 2003 \footnote{\url{http://www.sics.se/~joe/thesis/armstrong_thesis_2003.pdf}}
\end{itemize}

\section{Acknowledgments}

The first version of this document was based on a talk given by Mickael
R�mond\footnote{\email{mickael.remond@erlang-fr.org}} during an Object
Web benchmarking workshop in April 2004 (more info at
\url{http://jmob.objectweb.org/}).


\begin{appendix}

\section{Frequently Asked Questions}

\subsection{Tsung crashes when I start it }

Does your Erlang system has ssl support enabled ?

to test it:
\begin{Verbatim}
  > erl
  Eshell V5.2  (abort with ^G)
  1> ssl:start().
  you should see 'ok' 
\end{Verbatim}

\subsection{Tsung still doesn't start ...}

Most of the time, when a crash happened at startup without any traffic  
generated, the problem arise because the main Erlang controller node cannot  
create a "slave" Erlang virtual machine. The message looks like:

\begin{Verbatim}
===============================================
=ERROR REPORT==== 4-May-2004::22:38:26 ===
** Generic server ts_config_server terminating
** Last message in was {'$gen_cast',{newbeam,myshortname,[]}}
** When Server state == {state,{config,
                                     undefined,
                                     5,
                                     full,
                                     undefined,
                                     [{client,
                                          "myshortname",
                                          2.00000,
                                          5,
                                          [{10,68,133,140}]}],
                                     {server,"foo.net",80,gen_tcp},
                                     [],
                                     [{arrivalphase,
                                          1,
                                          60,
                                          undefined,
                                          undefined,
                                          5.00000e-5,
                                          infinity}],
                                     undefined,
                                     [{session,
                                          1,
                                          100,
                                          ts_http,
                                          parse,
                                          true,
                                          undefined}],
                                     14,
                                     3,
                                     7,
                                     6,
                                     "negociate"},
  "/home/username/.tsung/log/20040204-18:32",
                               undefined,
                               0,
                               undefined,
                               2.00000}
** Reason for termination ==
** {{badmatch,{error,timeout}},
    [{ts_config_server,handle_cast,2},
     {gen_server,handle_msg,6},
     {proc_lib,init_p,5}]}
\end{Verbatim} 
%%%$

Tsung launches a new erl virtual machine to do the actual
injection even when you have only one machine in the injection
cluster. This is because it needs to by-pass some limit with the number
of open socket from a single process (1024 most of the time). The idea
is to have several system processes (Erl beam) that can handle only a
small part of the network connection from the given computer. When the
\varname{maxclient} limit (simultaneous) is reach, a new Erlang beam is launched
and the newest connection can be handled by the new beam).

The problem is that the Erlang slave module cannot start a local slave
node. It tries to start it with the short node name
\varname{"myshortname"} (\command{erl -sname myshortname}).
 If this fails the injection process cannot
start. Most of the time, adding the short name with the correct IP
address in the \file{/etc/hosts} file is sufficient to make it work.

You can test this using these simple commands:
\begin{Verbatim}
>erl -rsh ssh -sname foo -setcookie mycookie

Eshell V5.4.3 (abort with ^G) 
(foo@myhostname)1>slave:start(remotehost,bar,"-setcookie mycookie").
{ok,bar@remotehost}
\end{Verbatim}

Note that you do not need to use the 127.0.0.1 address in the configuration file.  
It will not work if you use it as the injection interface. The shortname  
of your client machine should not refer to this address.

\strong{New in 1.1.0}: If you don't use the distributed feature of
Tsung and have trouble to start a remote beam on a local machine,
you can set the \varname{'use\_controller\_vm'} attribute to true, for ex.:

\begin{Verbatim}
  <client host="mymachine" use_controller_vm="true">
\end{Verbatim}

You may also have problems due to the sshd configuration:

For example, for SuSE 9.2 sshd is compiled with restricted set of
paths (\ie{} when you shell into the account you get the users shell,
when you execute a command via ssh you don't) and this makes it
impossible to start an erlang node (if erlang is installed in
\file{/usr/local} for example).

Run:

\begin{Verbatim}
ssh myhostname erl
\end{Verbatim}

If the erlang shell doesn't start then check what paths sshd was compiled with 
(in SuSE see \file{/etc/ssh/sshd_config}) and symlink from one of the approved paths 
to the erlang executable (thanks to Gordon Guthrie for reporting this).

\subsection{Tsung still crashes/fails  when I start it !}
First look at the log file
\file{~/.tsung/log/XXX/tsung_controller@yourhostname'} to see
if there is a problem. 

If the file is not created and a crashed dump file is present, maybe
you are using a binary installation of Tsung not compatible with the
version of erlang you used.

If you see nothing wrong, you can compile \program{Tsung} with full
debugging: recompile with \command{make debug} , and
don't forget to set the loglevel to "debug" in the XML file.

To start the debugger or see what happen, start \program{tsung} with the
\userinput{debug} argument instead of \userinput{start}. You will have
an erlang shell on the \varname{tsung\_controller} node. Use
\command{toolbar:start().} to launch the graphical tools provided by
Erlang.
\subsection{What is the format of the stats file tsung.log ?}

\begin{Verbatim}
# stats: dump at 1083694995
stats: users 11 11
stats: request 41 1.03289 0.125108 1.59802 0.901978
stats: connect 41 0.220170 6.67110e-2 0.494019 0.171997
stats: users_count 11 11
stats: page 24 6.80416 17.2794 80.4609 0.958984
stats: size 26818 26818
stats: 404 7 7
stats: 200 20 20
# stats: dump at 1083695005
stats: users 21 21
stats: request 113 1.03980 0.110650 1.59802 0.791016
stats: connect 118 0.197619 4.26037e-2 0.494019 0.163940
stats: users_count 10 21
stats: page 52 2.72266 1.74204 80.4609 0.791016
stats: size 78060 104878
stats: 404 15 22
stats: 200 51 71
 ...
\end{Verbatim}

 the format is, for \varname{request}, \varname{page}, \varname{session}:
 
 \texttt{ \# stats:'name' count(during the last 10sec), mean, stdvar,
   max, min}

 or for HTTP returns code, size ...

\texttt{ \# stats:'name' count(during the last 10sec), totalcount(since the beginning)}

\subsection{How can I specify the number of concurrent users ?}

You can't. But it's on purpose: the load generated by
\program{Tsung} is dependent on the arrival time between new
clients. Indeed, once a client has finished his session in
\program{tsung}, it stops. So the number of concurrent users is
a function of the arrival rate and the mean session duration.

For example, if your web site has $1000$ visits/hour, the arrival rate
is $1000/3600 = 0.2778$ visits/second. If you want to simulate the same
load, set the inter-arrival time is to $1/0.27778 = 3.6 sec$ (\texttt{<users
interarrival="3.6" unit="second">} in the \varname{arrivalphase} node in the
XML config file).

\subsection{SNMP monitoring doesn't work ?!}

It use SNMP v1 and the 'public' community. It has been tested with
\url{http://net-snmp.sourceforge.net/}.

You can try with \command{snmpwalk} to see if your snmpd config is ok:

\begin{Verbatim}
>snmpwalk -v 1 -c public IP-OF-YOUR-SERVER .1.3.6.1.4.1.2021.4.5.0
UCD-SNMP-MIB::memTotalReal.0 = INTEGER: 1033436
\end{Verbatim}

There is a small bug in the \file{snmp\_mgr} module in old Erlang
release (R9C-0). You have to apply this patch to make it
work. This is fixed in erlang R9C-1 and up.


\begin{Verbatim}
--- lib/snmp-3.4/src/snmp_mgr.erl.orig  2004-03-22 15:21:59.000000000 +0100
+++ lib/snmp-3.4/src/snmp_mgr.erl       2004-03-22 15:23:46.000000000 +0100
@@ -296,6 +296,10 @@
     end;
 is_options_ok([{recbuf,Sz}|Opts]) when 0 < Sz, Sz =< 65535 ->
     is_options_ok(Opts);
+is_options_ok([{receive_type, msg}|Opts]) ->
+    is_options_ok(Opts);
+is_options_ok([{receive_type, pdu}|Opts]) ->
+    is_options_ok(Opts);
 is_options_ok([InvOpt|_]) ->
     {error,{invalid_option,InvOpt}};
 is_options_ok([]) -> true.
\end{Verbatim}

\end{appendix}

\end{document}


%%% for AucTex/Emacs : 
%%% Local Variables: 
%%% eval:(setenv "TEXINPUTS" ":.:~/cvs/projetdoc//common/styles:./images:./figures:")
%%% mode: latex
%%% End: 
